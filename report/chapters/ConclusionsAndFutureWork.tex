\chapter{Conclusions and future work}

\section{Conclusion}

The 5S Drifter project almost successfully demonstrated the feasibility of developing a low-power, low-cost, 
and autonomous oceanographic drifter for surface sea stream monitoring. This integrative effort combined
knowledge in embedded systems, wireless communication, sensor integration, and mechanical design to
deliver prototype capable of acquiring GPS position, sea surface temperature and IMU data.

The prototype, works in isolation, but it still needs polishing due to the problems in development.
The physical design of the drifter shell, inspired by successful CMEMS projects such as SONDA and NextSea,
will continue to be improved and tested in sea. 

Overall, the project represents a solid step toward scalable, autonomous marine data collection and 
demonstrates how embedded electronics can be applied effectively in oceanographic research. With future 
iterations and optimizations, the 5S Drifter could become a viable tool for environmental monitoring, 
coastal safety planning, and marine traffic management.


\section{Limitations and Future improvements}
Despite the functionality, the prototype has limitations. Most notably its dependence on battery replacement 
and inability to cover long deployment periods without manual intervention. These issues point to clear 
paths for future improvements, such as integrating solar charging and switching to more energy-efficient 
microcontrollers like the STM32L0 family. A potential full migration to an RTOS-based architecture would 
enhance multitasking and improve energy management.

\section{Final GIT}
\href{https://github.com/viniciuscacarvalho/5S}{Here} is the full project on github.

\section{Special Greetings}
At last, it's important to add the support from the 
CMEMS labs personal as well of the professor Tiago Matos and professor Carlos Faria for the
support with hardware selection and shell mechanical advise respectively.
        

\section{Prospect for future work}
